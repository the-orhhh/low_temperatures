\documentclass[a4paper, 10pt, conference]{ieeeconf}      % Use this line for a4
                                                          % paper

\IEEEoverridecommandlockouts                              % This command is only
                                                          % needed if you want to
                                                          % use the \thanks command
\overrideIEEEmargins
% See the \addtolength command later in the file to balance the column lengths
% on the last page of the document

\usepackage[utf8]{inputenc}
\usepackage[T1]{fontenc}
\usepackage{amsmath}
\usepackage{hyphenat}
\usepackage{graphicx}
\usepackage{comment}
\usepackage{fancyhdr}
\usepackage[backend=biber,sorting=none]{biblatex}
\addbibresource{citations.bib}

\title{Observation of Atmospheric Muons and Measurement of their Lifetime}

\author{Dolev Einav and Or Harpazi
% $^{1}$
\thanks{$^{1}$Raymond \& Beverly Sackler Faculty of Exact Sciences, Tel Aviv University, Israel
}
}

\pagestyle{fancy}
\headsep = 1cm

\setlength\headheight{38pt} %% just to make warning go away. Adjust the value after looking into the warning.

\begin{document}

\fancyhf{}


\fancyhead[L]{MEASUREMENT OF THE MUON LIFETIME}
\fancyhead[R]{TAU SCHOOL OF PHYSICS \& ASTRONOMY}

\maketitle




%%%%%%%%%%%%%%%%%%%%%%%%%%%%%%%%%%%%%%%%%%%%%%%%%%%%%%%%%%%%%%%%%%%%%%%%%%%%%%%%
\begin{abstract}

The process of detecting atmospheric muons and measuring their decay time, as observed inside a plastic scintillator, is presented. Atmospheric muons, produced by the decay of charged pions, decay inside a plastic scintillator according to the $\mu^-\to\nu_\mu+e^-+\bar\nu_e$ channel, as well as its charge conjugate; the resulting electron is detected following the initial detection of the passing muon, processing the two timestamps into measurements of the muon lifetime. A lifetime of $\tau= 2.062\pm0.050\ [\mu\text s]$ was measured and analysed.

\end{abstract}


%%%%%%%%%%%%%%%%%%%%%%%%%%%%%%%%%%%%%%%%%%%%%%%%%%%%%%%%%%%%%%%%%%%%%%%%%%%%%%%%
\section{INTRODUCTION}

lorem ipsum dolor sit amet.\cite{sample}


\appendix


\subsection{MONTE-CARLO SIMULATION}\label{appendix:mc}

\paragraph{bla bla bla}

\addtolength{\textheight}{-12cm}   % This command serves to balance the column lengths
                                  % on the last page of the document manually. It shortens
                                  % the textheight of the last page by a suitable amount.
                                  % This command does not take effect until the next page
                                  % so it should come on the page before the last. Make
                                  % sure that you do not shorten the textheight too much.

%%%%%%%%%%%%%%%%%%%%%%%%%%%%%%%%%%%%%%%%%%%%%%%%%%%%%%%%%%%%%%%%%%%%%%%%%%%%%%%%



%%%%%%%%%%%%%%%%%%%%%%%%%%%%%%%%%%%%%%%%%%%%%%%%%%%%%%%%%%%%%%%%%%%%%%%%%%%%%%%%



%%%%%%%%%%%%%%%%%%%%%%%%%%%%%%%%%%%%%%%%%%%%%%%%%%%%%%%%%%%%%%%%%%%%%%%%%%%%%%%%
%%%%%%%%%%%%%%%%%%%%%%%%%%%%%%%%%%%%%%%%%%%%%%%%%%%%%%%%%%%%%%%%%%%%%%%%%%%%%%%%




\printbibliography

\end{document}
